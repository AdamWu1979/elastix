\documentclass[]{article}

% Define page
\usepackage[top=1in, bottom=1in, left=1in, right=1in]{geometry}

% Include packages for text modification
\usepackage{amsmath,amsfonts,amssymb}
\usepackage{bm}

% Include packages for citing
\usepackage[noadjust]{cite}

% Include packages for graphics
\usepackage{graphicx,psfrag}
\usepackage{subfigure}

% Include packages for tables
\usepackage{array,multirow}
\usepackage{ctable}

% Define new commands
\newcommand{\etal}{\emph{et al.}}
\newcommand{\vT}{\bm{T}}
\newcommand{\vmu}{\bm{\mu}}
\newcommand{\vTmu}{\bm{T_{\mu}}}
\newcommand{\vTx}[1][]{\bm{T}_{#1}(\bm{x})}
\newcommand{\vTy}[1][]{\bm{T}_{#1}(\bm{y})}
\newcommand{\vx}[1][]{\bm{x}_{#1}}
\newcommand{\vxt}[1][]{\bm{\widetilde x}_{#1}}
\newcommand{\vy}[1][]{\bm{y}_{#1}}
\newcommand{\D}[2]{\frac{\partial #1}{\partial #2}}
\newcommand{\Dd}[3]{\frac{\partial^2 #1}{\partial #2 \partial #3}}
\newcommand{\elastix}{\texttt{elastix}}


%%%%%%%%%%%%%%%%%%%%%%%%%%%%%%%%%%%%%%%%%%%%%%%%%%%%%%%%%%%%%%%%%%%%%%%%%%%%%%%%%%%%%%%%%%%%

\begin{document}

\title{Spatial derivatives and penalty terms in ITK and \elastix}

\author{Marius Staring and Stefan Klein}
\date{}
\maketitle

%%%%%%%%%%%%%%%%%%%%%%%%%%%%%%%%%%%%%%%%%%%%%%%%%%%%%%%%%%%%%%%%%%%%%%%%%%%%%%%%%%%%%%%%%%%%
% main text

\section{Introduction}

Image registration is the process of aligning images, and can be
defined as an optimisation problem:
\begin{align}
\hat \vmu &= \arg \min_{\vmu} \mathcal{C}(I_F, I_M;
\vmu),\label{eq:reg}
\end{align}
with $I_F$ and $I_M$ the $d$-dimensional fixed and moving image,
respectively, and $\vmu$ the vector of parameters of size $N$ that
model the transformation $\vT$. The cost function $\mathcal{C}$
consists of a similarity measure $\mathcal{S}(I_F, I_M; \vmu)$ that
defines the quality of alignment. Examples are the mean square
difference, normalised correlation, and mutual information measure.
In order to regularise the transformation $\vT_{\vmu}$ often a
penalty term $\mathcal{P}(\vmu)$ is added to the cost function, so
it becomes:
\begin{align}
\mathcal{C} &= \alpha_1 \mathcal{S} + \alpha_2 \mathcal{P},
\end{align}
where $\alpha_1, \alpha_2$ user-defined constants that weigh
similarity against regularity.

Penalty term are often based on the first or second order spatial
derivatives of the transformation. For example the bending energy of
the transformation, which is arguably the most common penalty term,
is defined in 2D as:
\begin{align}
\mathcal{P}_{\mathrm{BE}}(\vmu) &= \frac{1}{P} \sum_{\vxt[i]}
\left\| \frac{\partial^2 \vT}{\partial \vx \partial \vx^T}(\vxt[i])
\right\|_F^2 \\
&= \frac{1}{P} \sum_{\vxt[i]} \sum_{j = 1}^2 \left( \frac{\partial^2
T_j}{\partial x_1^2}(\vxt[i]) \right)^2  + 2 \left( \frac{\partial^2
T_j}{\partial x_1 \partial x_2}(\vxt[i]) \right)^2 + \left(
\frac{\partial^2 T_j}{\partial x_2^2}(\vxt[i]) \right)^2,
\end{align}
where $P$ is the number of points $\vxt[i]$, and the tilde denotes
the difference between a variable and a given point over which a
term is evaluated.

The optimisation problem (\ref{eq:reg}) is frequently solved using
an iterative gradient descent routine:
\begin{align}
\vmu_{k+1} &= \vmu_k - a_k \frac{\partial \mathcal{C}}{\partial
\vmu} = \vmu_k - a_k \left( \alpha_1 \frac{\partial
\mathcal{S}}{\partial \vmu} + \alpha_2 \frac{\partial
\mathcal{P}}{\partial \vmu} \right),\label{eq:opt}
\end{align}
with $a_k$ a user-defined declining function that defines the step
size.

The derivative of the similarity measure usually involves
computation of the spatial derivative of the moving image:
$\D{I_M}{\vx}$, and the derivative of the transformation to its
parameters: $\D{\vT}{\vmu}$. In the ITK the last derivative is
implemented using $\texttt{transform->GetJacobian()}$, i.e. the
derivative to the transformation parameters $\vmu$ is referred to as
`Jacobian'.

Penalty terms usually consist of the first and second order
\emph{spatial} derivatives of the transformation, i.e.
$\D{\vT}{\vx}$ and $\Dd{\vT}{\vx}{\vx^T}$. We will refer to these
derivatives as the `SpatialJacobian' and the `SpatialHessian' to
clearly distinguish between these derivatives and the `Jacobian'. In
order to apply the gradient descent optimisation routine
(\ref{eq:opt}), we additionally need the derivatives $\D{}{\vmu}
\D{\vT}{\vx}$ and $\D{}{\vmu} \Dd{\vT}{\vx}{\vx^T}$. These we call
the `JacobianOfSpatialJacobian' and `JacobianOfSpatialHessian',
respectively. See Table \ref{tab:notation} for details.

\begin{table}[h]
\centering
\begin{tabular}{llll}
\toprule \toprule
Name & definition & matrix size & written out in 2D \\
\midrule Transformation & $\vT = \vT_{\vmu}(\widetilde \vx)$ & $d
\times 1$ & $\begin{bmatrix} T_1(\vx) \\ T_2(\vx) \end{bmatrix}$ \\[3ex]
Jacobian & $\frac{\partial \vT_{\vmu}}{\partial \vmu}(\widetilde
\vx)$ & $d \times N$ & $\begin{bmatrix}
  \frac{\partial T_1}{\partial \mu_1}(\widetilde \vx) & \cdots & \frac{\partial T_1}{\partial \mu_N}(\widetilde \vx) \\
  \frac{\partial T_2}{\partial \mu_1}(\widetilde \vx) & \cdots & \frac{\partial T_2}{\partial \mu_N}(\widetilde \vx) \\
\end{bmatrix}$ \\[3ex]
SpatialJacobian & $\frac{\partial \vT_{\vmu}}{\partial
\vx}(\widetilde \vx)$ & $d \times d$ & $\begin{bmatrix}
  \frac{\partial T_1}{\partial x_1}(\widetilde \vx) \frac{\partial T_1}{\partial x_2}(\widetilde \vx) \\
  \frac{\partial T_2}{\partial x_1}(\widetilde \vx) \frac{\partial T_2}{\partial x_2}(\widetilde \vx) \\
\end{bmatrix}$ \\[3ex]
JacobianOfSpatialJacobian & $\frac{\partial}{\partial \vmu}
\frac{\partial \vT_{\vmu}}{\partial \vx}(\widetilde \vx)$ & $d
\times d \times N$ & $\begin{bmatrix}
  \frac{\partial}{\partial \mu_1} \frac{\partial \vT_{\vmu}}{\partial
\vx}(\widetilde \vx) & \cdots & \frac{\partial}{\partial \mu_N}
\frac{\partial \vT_{\vmu}}{\partial \vx}(\widetilde \vx)
\end{bmatrix}$ \\[3ex]
SpatialHessian & $\frac{\partial^2 \vT_{\vmu}}{\partial \vx
\partial \vx^T}$ & $d \times d \times d$ & $\left\{ \begin{bmatrix}
  \frac{\partial^2 T_1}{\partial x_1 \partial x_1}(\widetilde \vx) \frac{\partial^2 T_1}{\partial x_2 \partial x_1}(\widetilde \vx) \\
  \frac{\partial^2 T_1}{\partial x_2 \partial x_1}(\widetilde \vx) \frac{\partial^2 T_1}{\partial x_2 \partial x_2}(\widetilde \vx) \\
\end{bmatrix}, \begin{bmatrix}
  \frac{\partial^2 T_2}{\partial x_1 \partial x_1}(\widetilde \vx) \frac{\partial^2 T_2}{\partial x_2 \partial x_1}(\widetilde \vx) \\
  \frac{\partial^2 T_2}{\partial x_2 \partial x_1}(\widetilde \vx) \frac{\partial^2 T_2}{\partial x_2 \partial x_2}(\widetilde \vx) \\
\end{bmatrix} \right\}$ \\[3ex]
JacobianOfSpatialHessian & $\frac{\partial}{\partial \vmu}
\frac{\partial^2 \vT_{\vmu}}{\partial \vx \partial \vx^T}$ & $d
\times d \times d \times N$ & $\begin{bmatrix}
  \frac{\partial}{\partial \mu_1} \frac{\partial^2 \vT_{\vmu}}{\partial \vx
\partial \vx^T} & \cdots & \frac{\partial}{\partial \mu_N} \frac{\partial^2 \vT_{\vmu}}{\partial \vx
\partial \vx^T}
\end{bmatrix}$ \\
\bottomrule \bottomrule
\end{tabular}
\caption{Naming conventions and definitions for the transformation
and its derivatives used in this paper.}\label{tab:notation}
\end{table}

\section{Support for penalty terms in the ITK}

The derivative of the penalty term is not supported in the ITK. We
propose to add the following functions in the
\texttt{itk::Transform} class:
\begin{quote}
\begin{verbatim}
virtual void GetSpatialJacobian(
  const InputPointType &,
  SpatialJacobianType & ) const;

virtual void GetSpatialHessian(
  const InputPointType &,
  SpatialHessianType & ) const;

virtual void GetJacobianOfSpatialJacobian(
  const InputPointType &,
  JacobianOfSpatialJacobianType &,
  NonZeroJacobianIndicesType & ) const;

virtual void GetJacobianOfSpatialHessian(
  const InputPointType &,
  JacobianOfSpatialHessianType &,
  NonZeroJacobianIndicesType & ) const;
\end{verbatim}
\end{quote}
and additionally a function to implement a sparse version of the
Jacobian:
\begin{quote}
\begin{verbatim}
virtual void GetJacobian(
    const InputPointType &,
    JacobianType &,
    NonZeroJacobianIndicesType & ) const;
\end{verbatim}
\end{quote}

The ITK structures that were used to store the data are given in
Table \ref{tab:datastructures}. The Jacobian is of size $d \times
N$, and since the number of transformation parameters is flexible
for some transformations, the data structure used for storing the
Jacobian is an \texttt{itk::Array2D} object, which inherits from the
\texttt{vnl\_matrix}. This was already chosen previously in the ITK.
The SpatialJacobian is of fixed size $d \times d$, and therefore we
choose to use the \texttt{itk::Matrix} to store the SpatialJacobian,
which inherits from the \texttt{vnl\_matrix\_fixed}. For derivatives
to $\vmu$ we choose to use the \texttt{std::vector}. The
SpatialHessian gives us some problems, since we really need a 3D
matrix, but currently no such thing exists in the ITK. Therefore, we
opt for an \texttt{itk::FixedArray} of \texttt{itk::Matrix}.

\begin{table}[h]
\centering
\begin{tabular}{ll}
\toprule \toprule
Name & ITK structure \\
\midrule Transformation &  \\
Jacobian & \texttt{Array2D = vnl\_matrix} \\
SpatialJacobian & \texttt{Matrix = vnl\_matrix\_fixed} \\
JacobianOfSpatialJacobian & \texttt{std::vector< Matrix >} \\
SpatialHessian & \texttt{FixedArray< Matrix >}\footnotemark \\
JacobianOfSpatialHessian & \texttt{std::vector< FixedArray< Matrix > >} \\
NonZeroJacobianIndices & \texttt{std::vector< unsigned long >} \\
\bottomrule \bottomrule
\end{tabular}
\caption{The ITK structures that store the
data.}\label{tab:datastructures}
\end{table}
\footnotetext{by lack of a good 3D matrix structure}

From the function definitions above, notice that we chose to pass
the variables by reference, and as function arguments. The
\texttt{GetJacobian} in the \texttt{itk::Transform} is defined as:
\begin{quote}
\begin{verbatim}
virtual const JacobianType & GetJacobian( const InputPointType & ) const;
\end{verbatim}
\end{quote}
which returns a reference to the member variable
\texttt{m\_Jacobian}. It is possible, however, that this member is
only valid for the input point, namely for transformations with a
derivative dependent of the spatial position. Therefore, we think
that it should not be possible to access this parameter at a later
time, when the input point has possibly changed.

Also notice the \texttt{NonZeroJacobianIndicesType} in the function
definitions. These are meant for the support of sparse Jacobians,
JacobianOfSpatialJacobians, etc.

\section{Affine transformation}

For the affine transformation, the derivatives evaluate to the
following in 2D:
\begin{align}
\vTmu(\vxt) &= A \vx + \bm{t} = \begin{bmatrix} a_{11} & a_{12}
\\ a_{21} & a_{22} \end{bmatrix} \begin{bmatrix} \widetilde x_1 \\
\widetilde x_2 \end{bmatrix} + \begin{bmatrix} t_1 \\ t_2
\end{bmatrix} = \begin{bmatrix} \mu_0 & \mu_1
\\ \mu_2 & \mu_3 \end{bmatrix} \begin{bmatrix} \widetilde x_1 \\
\widetilde x_2 \end{bmatrix} + \begin{bmatrix} \mu_4 \\ \mu_5
\end{bmatrix}, \\
\D{\vT}{\vmu}(\vxt) &= \begin{bmatrix} \widetilde x_1 & \widetilde
x_2 & 0 & 0 & 1 & 0 \\ 0 & 0 & \widetilde x_1 & \widetilde x_2 & 0 &
1 \end{bmatrix}, \\
\D{\vT}{\vx}(\vxt) &= \begin{bmatrix} \mu_0 & \mu_1 \\ \mu_2 & \mu_3 \end{bmatrix}, \\
\Dd{\vT}{\vx}{\vx^T}(\vxt) &= O_{d \times d \times d}, \\
\D{}{\vmu} \D{\vT}{\vx}(\vxt) &= \left\{ \begin{bmatrix} 1 & 0 \\
0 & 0 \end{bmatrix}, \begin{bmatrix} 0 & 1 \\ 0 & 0
\end{bmatrix}, \begin{bmatrix} 0 & 0 \\ 1 & 0
\end{bmatrix}, \begin{bmatrix} 0 & 0 \\ 0 & 1
\end{bmatrix} \right\}, \\
\D{}{\vmu} \Dd{\vT}{\vx}{\vx^T}(\vxt) &= O_{d \times d \times d
\times N},
\end{align}
where $O_s$ is a zero matrix of size $s$.

%centre of rotation!!!

The \texttt{GetJacobianOfSpatialJacobian()} returns
\texttt{nonZeroJacobianIndices} = $[ 0, 1, 2, 3 ]$.

%For the SpatialHessian and the JacobianOfSpatialHessian we choose to
%not return matrices filled with zeros, but instead zero sized
%matrices. The implementation of the penalty term is assumed to check
%for the size. In case of an affine transform the size is zero, and
%the penalty term can simply return zero. This is a performance
%benefit compared to walking over the zero matrix, and adding and
%multiplying everything, which in the end also gives zero.

These derivatives are implemented in the
\texttt{itk::AdvancedMatrixOffsetTransformBase} class.

\section{B-spline transformation}

A transformation parameterised by third order B-splines can be
written in 2D as follows:
\begin{align}
\vTmu(\vxt) &= \begin{bmatrix} T_1(\vxt;\vmu) \\ T_2(\vxt;\vmu)
\end{bmatrix} = \begin{bmatrix} \widetilde{x}_1 \\ \widetilde{x}_2 \end{bmatrix}
 + \begin{bmatrix} \sum_{i} \mu_i \beta^3 \left(
\frac{ \widetilde{x}_1 - x_{1}^i}{\sigma_1} \right)
\beta^3 \left( \frac{\widetilde{x}_2 - x_{2}^i}{\sigma_2}\right) \\
 \sum_{i} \mu_{i+16} \beta^3 \left( \frac{ \widetilde{x}_1 - x_{1}^i}
 {\sigma_1} \right) \beta^3 \left( \frac{\widetilde{x}_2 - x_{2}^i}{\sigma_2} \right)
\end{bmatrix},
\end{align}
with $\vx^i$ the control points within the support of the B-spline
basis functions $\beta^3(\cdot)$, and $\sigma$ the B-spline grid
spacing. Here, we labelled the $x$-direction of the control points
with $\mu_0, \ldots, \mu_{15}$ and the $y$-direction with $\mu_{16},
\ldots, \mu_{31}$. (Since the support of a cubic B-spline is $4^d =
16$.)

For short notation, define:
\begin{align}
b_{33}^{i} &= \beta^3\left( (\widetilde{x}_1 - x_{1}^i) / \sigma_1
\right) \cdot \beta^3 \left( (\widetilde{x}_2 - x_{2}^i) / \sigma_2
\right), \\
b_{23}^{i} &= \left[ \beta^2 \left( (\widetilde{x}_1 -
x_{1}^i)/\sigma_1 + \tfrac{1}{2} \right) - \beta^2 \left(
(\widetilde{x}_1 - x_{1}^i)/\sigma_1 - \tfrac{1}{2} \right) \right]
\cdot \beta^3 \left( (\widetilde{x}_2 - x_{2}^i)/\sigma_2 \right) /
\sigma_1, \\
b_{32}^{i} &= \beta^3 \left( (\widetilde{x}_1 - x_{1}^i)/\sigma_1
\right) \cdot \left[ \beta^2 \left( (\widetilde{x}_2 -
x_{2}^i)/\sigma_2 + \tfrac{1}{2} \right) - \beta^2 \left( (
\widetilde{x}_2 - x_{2}^i)/\sigma_2 - \tfrac{1}{2} \right) \right] /
\sigma_2, \\
\begin{split}
b_{22}^{i} &= \left[ \beta^2 \left( ( \widetilde{x}_1 -
x_{1}^i)/\sigma_1 + \tfrac{1}{2} \right) - \beta^2 \left( (
\widetilde{x}_1 - x_{1}^i)/\sigma_1 - \tfrac{1}{2} \right) \right] \\
& \qquad \qquad \cdot \left[ \beta^2 \left( ( \widetilde{x}_2 -
x_{2}^i)/\sigma_2 + \tfrac{1}{2} \right) - \beta^2 \left( (
\widetilde{x}_2 - x_{2}^i)/\sigma_2 - \tfrac{1}{2} \right) \right] /
(\sigma_1 \sigma_2),
\end{split} \\
b_{13}^{i} &= \left[ \beta^1 \left( (\widetilde{x}_1 -
x_{1}^i)/\sigma_1 + 1 \right) - 2 \beta^1 \left( ( \widetilde{x}_1 -
x_{1}^i)/\sigma_1 \right) + \beta^1 \left( (\widetilde{x}_1 -
x_{1}^i)/\sigma_1 - 1 \right) \right] \cdot \beta^3 \left(
(\widetilde{x}_2 - x_{2}^i)/\sigma_2 \right) / \sigma_1^2, \\
b_{31}^{i} &= \beta^3 \left( (\widetilde{x}_1 - x_{1}^i)/\sigma_1
\right) \cdot \left[ \beta^1 \left( (\widetilde{x}_2 -
x_{2}^i)/\sigma_2 + 1 \right) - 2 \beta^1 \left( ( \widetilde{x}_2 -
x_{2}^i)/\sigma_2 \right) + \beta^1 \left( (\widetilde{x}_2 -
x_{2}^i)/\sigma_2 - 1 \right) \right] / \sigma_2^2.
\end{align}
From these equations we derive:
\begin{align}
\D{\vT}{\vmu}(\vxt) &= \begin{bmatrix} b_{33}^{0} & \cdots &
b_{33}^{15} & 0 & \cdots & 0 \\ 0 & \cdots & 0 & b_{33}^{0} &
\cdots & b_{33}^{15} \end{bmatrix}, \\
\D{\vT}{\vx}(\vxt) &= \begin{bmatrix} 1 + \sum_i \mu_i b_{23}^i &
\sum_i \mu_i b_{32}^i \\ \sum_i \mu_{i+16} b_{23}^{i} & 1 + \sum_i
\mu_{i+16} b_{32}^{i} \end{bmatrix}, \\
\Dd{\vT}{\vx}{\vx^T}(\vxt) &= \left\{ \begin{bmatrix} \sum_i \mu_i
b_{13}^i & \sum_i \mu_i b_{22}^i \\ \sum_i \mu_{i} b_{22}^{i} &
\sum_i \mu_{i} b_{31}^{i} \end{bmatrix}, \begin{bmatrix} \sum_i
\mu_{i+16} b_{13}^{i} & \sum_i \mu_{i+16} b_{22}^{i} \\ \sum_i
\mu_{i+16} b_{22}^{i} & \sum_i \mu_{i+16} b_{31}^{i}
\end{bmatrix} \right\} \\
\D{}{\vmu} \D{\vT}{\vx}(\vxt) &= \left\{ \begin{bmatrix} b_{23}^{0}
& b_{32}^{0} \\ 0 & 0 \end{bmatrix}, \cdots, \begin{bmatrix}
b_{23}^{15} & b_{32}^{15} \\ 0 & 0 \end{bmatrix}, \begin{bmatrix} 0
& 0 \\ b_{23}^{0} & b_{32}^{0} \end{bmatrix}, \cdots,
\begin{bmatrix} 0 & 0 \\ b_{23}^{15} & b_{32}^{15} \end{bmatrix} \right\}
\\
\D{}{\vmu} \Dd{\vT}{\vx}{\vx^T}(\vxt) &= \left\{ \left\{
\begin{bmatrix} b_{13}^0 & b_{22}^0 \\ b_{22}^0 & b_{31}^0 \end{bmatrix},
O_{d \times d} \right\}, \cdots, \left\{ \begin{bmatrix} b_{13}^{15}
& b_{22}^{15} \\ b_{22}^{15} & b_{31}^{15} \end{bmatrix}, O_{d
\times d} \right\}, \right. \\
& \qquad \left. \left\{ O_{d \times d},
\begin{bmatrix} b_{13}^{0} & b_{22}^{0} \\ b_{22}^{0} &
b_{31}^{0} \end{bmatrix} \right\}, \cdots, \left\{ O_{d \times d},
\begin{bmatrix} b_{13}^{15} & b_{22}^{15} \\ b_{22}^{15} &
b_{31}^{15} \end{bmatrix} \right\} \right\}.
\end{align}

These derivatives are implemented in the
\texttt{itk::AdvancedBSplineDeformableTransform} class.

\section{Combining transformations}

\texttt{elastix} supports combining multiple transformations by
addition or composition. Adding transformations is done via:
\begin{align}
\vTx &= \vTx[0] + \vTx[1] - \vx,
\end{align}
where $\vTx[0]$ is the initial transformation and $\vTx[1]$ the
current transformation. Only the current transformation is optimised
during the registration (as a choice). Composition of
transformations is defined via:
\begin{align}
\vTx &= \vT_1(\vTx[0]).
\end{align}

For these combined transformations we need to derive the relations
for the derivatives. Define $\bm{y} = \vT_0(\vxt)$.

\begin{align}
\intertext{Jacobian:}
\D{\vTmu}{\vmu}(\vxt) &= \D{}{\vmu} \left( \vT_0(\vxt) + \vT_1(\vxt)
- \vxt \right) = \D{}{\vmu} \vT_1(\vxt), \\
\D{\vTmu}{\vmu}(\vxt) &= \D{}{\vmu} \left( \vT_1(\vT_0(\vxt))
\right) = \D{}{\vmu} \vT_1(\bm{y}).
\intertext{SpatialJacobian:}
\D{\vTmu}{\vx}(\vxt) &= \D{}{\vx} \left( \vT_0(\vxt) + \vT_1(\vxt) -
\vxt \right) = \D{}{\vx} \vT_0(\vxt) + \D{}{\vx} \vT_1(\vxt) - \bm{I}, \\
%\D{\vTmu}{\vx}(\vxt) &= \D{}{\vx} \vT_1(\vT_0(\vxt)) = \D{}{\vx}
%\vT_1(\bm{y}) \cdot \D{}{\vx} \vT_0(\vxt). \\
\D{T_{\vmu,k}}{x_i}(\vxt) &= \left( \D{T_{1,k}}{\vx}(\vy) \right)^T
\D{\vT_0}{x_i}(\vxt) =  \left(\D{\vT_0}{x_i}(\vxt) \right)^T
\D{T_{1,k}}{\vx}(\vy)
\intertext{JacobianOfSpatialJacobian:}
\D{}{\vmu} \D{\vTmu}{\vx}(\vxt) &= \D{}{\vmu} \D{}{\vx} \left(
\vT_0(\vxt) + \vT_1(\vxt) - \vxt \right) = \D{}{\vmu} \D{}{\vx}
\vT_1(\vxt), \\
\D{}{\vmu} \D{\vTmu}{\vx}(\vxt) &= \D{}{\vmu} \D{}{\vx} \left(
\vT_1(\vTx[0]) \right) = \D{}{\vmu} \D{}{\vx} \vT_1(\bm{y}) \cdot
\D{}{\vx} \vT_1(\bm{y})\vT_0(\vxt).
\intertext{SpatialHessian:}
\Dd{\vTmu}{x_i}{x_j} &= \Dd{}{x_i}{x_j} \left( \vT_0(\vxt) +
\vT_1(\vxt) \right) = \Dd{}{x_i}{x_j} \vT_0(\vxt) +
\Dd{}{x_i}{x_j} \vT_1(\vxt), \\
%\Dd{\vTmu}{x_i}{x_j} &= \Dd{}{x_i}{x_j} \left( \vT_1(\vT_0(\vxt))
%\right) = \D{}{x_i} \vT_0(\vxt) \Dd{}{x_i}{x_j} \vT_1(\bm{y})
%\D{}{x_j} \vT_0(\vxt) + \Dd{}{x_i}{x_j} \vT_0(\vxt) \D{}{x_j}
%\vT_1(\bm{y}). \\
\Dd{T_{\vmu,k}}{x_i}{x_j}(\vxt) &= \left( \D{T_{1,k}}{\vx}(\vy)
\right)^T \Dd{\vT_0}{x_i}{x_j}(\vxt) + \left(\D{\vT_0}{x_i}(\vxt)
\right)^T \Dd{T_{1,k}}{\vx}{\vx^T}(\bm{y}) \D{\vT_0}{x_j}(\vxt)
\intertext{JacobianOfSpatialHessian:}
\D{}{\vmu} \Dd{\vTmu}{x_i}{x_j} &= \D{}{\vmu} \Dd{}{x_i}{x_j} \left(
\vT_0(\vxt) + \vT_1(\vxt) - \vxt \right) = \D{}{\vmu}
\Dd{}{x_i}{x_j} \vT_1(\vxt), \\
%\D{}{\vmu} \Dd{\vTmu}{\vx}{\vx^T} &= \D{}{\vmu} \Dd{}{x_i}{x_j}
%\left( \vT_1(\vT_0(\vxt)) \right) \\
% &= \D{}{x_i} \vT_0(\vxt) \D{}{\vmu} \Dd{}{x_i}{x_j} \vT_1(\bm{y})
%\D{}{x_j} \vT_0(\vxt) + \Dd{}{x_i}{x_j} \vT_0(\vxt) \D{}{\vmu}
%\D{}{x_j} \vT_1(\bm{y}). \\
\D{}{\vmu} \Dd{T_{mu,k}}{x_i}{x_j}(\vxt) &=
\Dd{T_{1,k}}{\vmu}{\vx^T}(\vy) \Dd{\vT_0}{x_i}{x_j}(\vxt) +
\left(\D{\vT_0}{x_i}(\vxt) \right)^T \left(\D{}{\vmu}
\Dd{T_{1,k}}{\vx}{\vx^T}(\bm{y})\right) \D{\vT_0}{x_j}(\vxt)
\end{align}

These derivatives are implemented in the
\texttt{itk::AdvancedCombinationTransform} class.

\section{Bending energy penalty term}

The bending energy was defined as:
\begin{align}
\mathcal{P}_{\mathrm{BE}}(\vmu) &= \frac{1}{P} \sum_{\vxt[i]}
\sum_{k,l,m = 1}^2 \left( \Dd{T_k}{x_l}{x_m}(\vxt[i]) \right)^2.
\end{align}

We constructed an \texttt{itk::TransformBendingEnergyPenaltyTerm}
which inherits from the \texttt{itk::ImageToImage\-Metric}. The
\texttt{GetValue()}-method is implemented like:
\begin{quote}
\begin{verbatim}
SpatialHessianType spatialHessian; for all samples
  this->GetTransform()->GetSpatialHessian( sample, spatialHessian );
  for all k, l, m
    measure += spatialHessian[ k ][ l ][ m ]^2;
  end
end measure /= this->m_NumberOfPixelsCounted;
\end{verbatim}
\end{quote}

For the \texttt{GetValueAndDerivative()} we have:
\begin{align}
\D{}{\vmu} \mathcal{P}_{\mathrm{BE}}(\vmu) &=  \frac{1}{P}
\sum_{\vxt[i]} \sum_{k,l,m = 1}^2 2 \Dd{T_k}{x_l}{x_m} (\vxt[i])
\D{}{\vmu} \Dd{T_k}{x_l}{x_m}(\vxt[i]),
\end{align}
which is implemented like:
\begin{quote}
\begin{verbatim}
SpatialHessianType spatialHessian; JacobianOfSpatialHessianType
jacobianOfSpatialHessian; for all samples
  this->GetTransform()->GetSpatialHessian( sample, spatialHessian );
  this->GetTransform()->GetJacobianOfSpatialHessian( sample,
    jacobianOfSpatialHessian, nonZeroJacobianIndices );
  for all nonZeroJacobianIndices, k, l, m
    derivative[ nonZeroJacobianIndices[ mu ] ] += 2.0
      * spatialHessian[ k ][ i ][ j ]
      * jacobianOfSpatialHessian[ mu ][ k ][ i ][ j ];
  end
end derivative /= this->m_NumberOfPixelsCounted;
\end{verbatim}
\end{quote}

This penalty term is implemented in the
\texttt{itk::TransformBendingEnergyPenaltyTerm} class.

%%%%%%%%%%%%%%%%%%%%%%%%%%

\section{Discussion and Conclusion}

This document describes the use and implementation of spatial
derivatives of coordinate transformations. These spatial derivatives
were exploited by the bending energy penalty term. Their usage is,
however, not limited to that penalty term, and many more penalty
terms can be implemented using the new functionality.

%%%%%%%%%%%%%%%%%%%%%%%%%%%%%%%%%%%%%%%%%%%%%%%%%%%%%%%%%%%%%%%%%%%%%%%%%%%%%%%%%%%%%%%%%%%%

%\bibliographystyle{plainnat}
%\bibliography{mi}

%%%%%%%%%%%%%%%%%%%%%%%%%%%%%%%%%%%%%%%%%%%%%%%%%%%%%%%%%%%%%%%%%%%%%%%%%%%%%%%%%%%%%%%%%%%%

\end{document}
